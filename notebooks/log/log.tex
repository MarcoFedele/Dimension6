\documentclass[a4paper,12pt]{article}
\usepackage{geometry}
\usepackage{fancyhdr}
\usepackage{amsmath,amsthm,amssymb, mathrsfs}
\usepackage{graphicx, subfigure}
\usepackage{hyperref}
\usepackage[italian]{babel}
\usepackage{braket}
\usepackage{graphicx}
\usepackage{slashed}
\usepackage{cite}
\usepackage{multirow}
\usepackage{caption}
\usepackage[dvipsnames]{xcolor}
\usepackage{listings}
\usepackage{color}
\usepackage[utf8x]{inputenc}
\usepackage{listings}


\newcommand\m{\mu}
\newcommand\la{\lambda}
\newcommand\g{\gamma}
\newcommand\s{\sigma}
\newcommand\al{\alpha}
\newcommand\e{\epsilon}
\newcommand\thet{\theta}
\newcommand\SI{\Sigma}
\newcommand\DE{\Delta}
\newcommand\LA{\Lambda}

\newcommand\va{\rightarrow}
\newcommand\re{\operatorname{Re}}
\newcommand\im{\operatorname{Im}}
\newcommand{\cor}[1]{\mathscr{#1}}
\newcommand\ov[1]{\frac{1}{#1}}
\newcommand\de{\partial}
\newcommand\imp{\Rightarrow}
\newcommand\av[1]{\left\langle #1\right\rangle}
\newcommand{\beq}{\begin{equation}}
\newcommand{\eeq}{\end{equation}}
\newcommand{\bit}{\begin{itemize}}
\newcommand{\eit}{\end{itemize}}
\newcommand{\itm}{\item[-]}
\newcommand{\bea}{\begin{eqnarray}}
\newcommand{\eea}{\end{eqnarray}}
\newcommand\li{\mathrm{Li_2}}
\newcommand\Var[1]{\mathrm{Var}(#1)}
\newcommand\Cov[1]{\mathrm{Cov}(#1)}


\begin{document} 


\section*{1301.2588, \url{http://arxiv.org/pdf/1301.2588v1.pdf}}
Dimensione anomala degli operatori $O_{HB,HW,HWB,HG}$ e $\tilde O_{HB,HW,HWB,HG}$: calcolo ultimato.\\
\emph{Note}:
\bit
\itm calcolo svolto nella gauge di Feynman;
\itm nella definizione degli operatori è presente la costante $g_i^2/2$;
\itm i bosoni vettori sono scelti come campi entranti, e i loro indici di Lorentz vengono contratti nel diag, attraverso la seguete espressione; \\
\begin{lstlisting}%
[frame=trbl]
exp = Expand[ exp FourVector[Ep1, Index[Lorentz,1]] *
               FourVector[Ep2, Index[Lorentz,2]] ];
\end{lstlisting}
la conseguenza di questa contrazione è che bisogna elimare i seguenti termini spuri: \\
\begin{lstlisting}%
[frame=trbl]
exp =  exp /. {sp[Ep1,q1] -> 0, sp[Ep2,q2] -> 0};
\end{lstlisting}
\itm i bosoni vettori sono a massa nulla: bisogna quindi porre \\
\begin{lstlisting}%
[frame=trbl]
exp =  exp /. {sp[q1,q1] -> 0, sp[q2,q2] -> 0};
\end{lstlisting}
di conseguenza, le $B_0$ in cui la particella del loop è massless e che hanno come unica scala il momento esterno di uno dei bosoni vettori sono nulle. Questa eliminazione viene eseguita nel \emph{pp0simplify}, in cui vengono rimosse non solo le $A_0$ massless ma anche le $B_0$ massless in cui l'unico momento presente è $q1$ o $q2$;
\itm il contributo della wavefunction dell'higgs viene aggiunto a mano, dopo aver caricato il file H\_WF.res .
\eit



\newpage



\section*{1308.2627, \url{http://arxiv.org/pdf/1308.2627v3.pdf}}
Dimensione anomala di tutti gli operatori, termini proporzionali a $\la$, $\la^2$ e $\la y^2$. Due tipi di contributi: quelli diretti, e quelli indiretti ottenuti tramite l'applicazione delle equazioni del moto (stile operatori pinguino). Per quanto riguarda i termini indiretti, ho proceduto calcolando i diagrammi 1PR e considerando la parte non dipendente dal propagatore esterno al loop.
\subsection*{Classe $H^6$}
Calcolo ultimato, incongruenze numeriche relativamente agli opertori della classe $H^4D^2$. Non sono stati (ancora) calcolati i diagrammi 1PR data l'elevata complicazione le ridurre il risultato finale ad una forma priva di sp.
\subsection*{Classe $X^2H^2$}
Calcolo ultimato.
\subsection*{Classe $H^4D^2$}
Calcolo ultimato. Questi operatori mixano anche con il quartico dello SM $O_{\la}$, quindi sono necessari alcuni accorgimenti per ottenere il risultato esatto:
\bit
\itm per quanto riguarda l'operatore $O_{HD}$, si è sfruttato il fatto che il vertice $h\phi^0\phi^+\phi^-$ non fosse presente in $O_{\la}$;
\itm per quanto riguarda l'operatore $O_{H\Box}$, si sono sfruttate considerazioni cinematiche ed equazioni del moto del tipo
\beq
h = q_1\cdot q_1 = q_1\cdot q_3 + q_1\cdot q_4 - q_1\cdot q_2 = 3h -2q_2\cdot q_3-2q_2\cdot q_4+2q_3\cdot q_4  \nonumber
\eeq
per riuscire a scrivere il risultato come combinazione lineare dei soli operatori  $O_{H\Box}$ e  $O_{\la}$.
\eit
Per entrambi gli operatori, si è eseguito il calcolo per tutti i possibili vertici, assicurandosi di ottenere sempre lo stesso risultato.
\subsection*{Classe $\psi^2H^3$}
Calcolo ultimato, incongruenze numeriche relativamente agli opertori della classe $H^4D^2$: risultati diversi a seconda del vertice considerato, in molti dei vertici non si riescono a rimuovere tutti gli sp (che non sono presenti nella struttura di lorentz del vertice da rinormalizzare, e sembrano quindi spuri).



\newpage



\section*{1310.4838, \url{http://arxiv.org/pdf/1310.4838v2.pdf}}
Dimensione anomala di tutti gli operatori, termini proporzionali a $y^i$. Due tipi di contributi: quelli diretti, e quelli indiretti ottenuti tramite l'applicazione delle equazioni del moto (stile operatori pinguino). Per quanto riguarda i termini indiretti, ho proceduto calcolando i diagrammi 1PR e considerando la parte non dipendente dal propagatore esterno al loop.
\subsection*{Classe $H^6$}
Calcolo ultimato, fattore 3/4 overall.
\subsection*{Classe $H^4D^2$}
Calcolo ultimato.
\subsection*{Classe $X^2H^2$}
Calcolo ultimato, fattore 2 overall (tranne che in $O_{HG}$), incapacità nel distinguere i contributi per $O_X$ da quelli per $\tilde O_X$. Risultato immaginario, come dovrebbe essere quello per $\tilde O_X$ ma non quello per $O_X$.
\subsection*{Classe $\psi^2H^3$}
Calcolo ultimato solo per 1PI, varie discrepanze numeriche.
\subsection*{Classe $\psi^2XH$}
Risultato sempre pari a 0, probabilmente dovuto all'incapacità di identificare correttamente la struttura di Lorentz dell'operatore.
\subsection*{Classe $\psi^2H^2D$}
Calcolo ultimato solo per 1PI, varie discrepanze numeriche.
\subsection*{Classe $\psi^4$}
Calcolo ultimato solo per 1PI, con le seguenti (relativamente) poche discrepanze numeriche: fattore 2 davanti a tutti i contributi di $O_{lequ}^{(3)}$, fattore 2 in quasi tutti gli elementi di $O_{le}$, $O_{qu}^{(1,8)}$, $O_{qd}^{(1,8)}$ e $O_{quqd}^{(1,8)}$, impossibilità di riprodurre/distinguere $O_{lequ}^{(3)}$ da $O_{lequ}^{(1)}$. \\
\emph{Note}:
\bit
\itm calcolo svolto differenziando esplicitamente i campi L dai campi R. Questa operazione ha avuto come conseguenza il fatto che ogni vertice $\psi^4$ venisse effettivamente contato 4 volte da FeynRules: il risultato finale viene quindi ottenuto dividendo il risultato del calcolo per 4;
\itm negli operatori tipo $O_{ee}^{jklm}$, dove tutti i campi sono uguali, vale la seguente equivalenza tra operatori con indici di flavor espliciti:
\beq
O_{ee}^{1133}=O_{ee}^{3311},\qquad O_{ee}^{1331}=O_{ee}^{3113} \nonumber
\eeq
occorre dunque dividere ulteriormente per 2, per ottenere il risultato corretto.
\eit





\newpage


\section*{Compendio}
In questa sezione vengono riportate le funzioni introdotte, il loro ruolo e la logica di base.
\bit
\itm {\bf ComputeCheckdiag}\\ Conta il numero di termini presenti per ogni diagramma, restituendo una lista ordinata di 1 (per ogni diagramma formato da più termini) e di 0 (per ogni diagramma formato da un singolo termine).
\itm {\bf ppcounter}\\ Conta il numero di pp[\_\_\_] presenti in ogni termine di ogni diagramma. Esempio: \\
\begin{lstlisting}%
[frame=trbl]
{
{2,2,2,2},        <-- I diag, 4 term
{2,2},            <-- II diag, 2 term
{2,2,3,3,2,3}     <-- III diag, 6 term
}
\end{lstlisting}
\itm {\bf spcounter}\\ Conta il numero di sp[\_\_\_] presenti in ogni termine di ogni diagramma. Esempio:  \\
\begin{lstlisting}%
[frame=trbl]
{
{2,1,1,1},        <-- I diag, 4 term
{0,0},            <-- II diag, 2 term
{2,1,2,2,1,0}     <-- III diag, 6 term
}
\end{lstlisting}
\itm {\bf ppspcounter}\\ Crea una lista in cui viene inserito \{pp,sp\} per ogni termine di ogni diagramma. Esempio:  \\
\begin{lstlisting}%
[frame=trbl]
{
{ {2,2},{2,1},{2,1},{2,1} },  <-- I diag, 4 term
{ {2,0},{2,0} },              <-- II diag, 2 term
{ {2,2},{2,1},{3,2},{3,2},{2,1},{3,0} } <-- III d, 6 t
}
\end{lstlisting}
\itm {\bf pp0simplify}\\ Rimuove i diagrammi senza scala, e quindi nulli, in regolarizzazione dimensionale.
\itm {\bf diagSimplify}\\ Semplifica i diagrammi applicando, nell'ordine, ppSimplify,  ppAbsorbMomenta, ExpandScalarProducts e pp0simplify.
\itm {\bf PV}\\ Esegue la riduzione ad integrali scalari con il metodo Passerino-Veltmann, con il seguente procedimento:
\bit
\itm viene eseguito ComputeCheckdiag; a seconda del risultato, ci si concentrerà nelle successive iterazioni su ppspcounter[[DIAG,TERM]] o su ppspcounter[[DIAG,1]];
\itm si legge il valore di ppspcounter;
\itm per \{0,0\}, non si fa nulla;
\itm per \{1,0\},\{2,0\},\{3,0\} e \{4,0\} si scrive il risultato in termini delle fuonzioni PV;
\itm per gli altri casi si eseguono prima le sostituzioni ( /. ruleXY ), poi le semplificazioni ( diagSimplify ), e infine si scrive il risultato in termini delle funzioni PV; da notare che prima di rule21 e rule22 viene eseguito uno shift per essere sicuri che a denominatore ci sia almeno un'espressione della forma pp[p, \_ ]; infine, nel caso \{3,2\}, vengono applicate sia le rule3X che le rule2X, intervallate da un diagSimplify.
\itm ogni risultato temporale (chiamato temp1) viene appeso alla lista temp0, che verrà infine spacchettata tramite l'uso di ResumDiag; nei casi in cui sp è almeno 1, viene appeso temp1[[1]]
\eit
\eit
Il modo corretto per chiamarla su un singolo diagramma costituito rispettivamente da un singolo termine o da più termini è il seguente: PV[\{diag\},\{\{\{pp,sp\}\}\}] o PV[\{diag\},\{\{\{pp$_1$,sp$_1$\},\{pp$_2$,sp$_2$\},\dots\}\}].










\end{document}

